\documentclass[]{article} %% skriv in 12pt om ni vill göra textstorleken större

\usepackage[margins=2cm]{geometry} % Sidstorlek och sidmarginaler
\usepackage[utf8]{inputenc} % Tillhör LaTeX
\usepackage[T1]{fontenc} % Tillhör LaTeX
\usepackage{ae} % Tillhör LaTeX
\usepackage[english,swedish]{babel} % Andra språket är det som granskas med avseende på stavning
\usepackage{amsmath} % Matematisk
\usepackage{amssymb} % Matematisk
\usepackage{amsfonts} % Matematisk
\usepackage{bbm} % Matematisk
\usepackage{units} % Enheter
\usepackage{mhchem} % Kemiska formler
\usepackage{icomma} % Ett intelligent komma, som ger rätt avstånd i decimaltal och även i  matematiska uttryck.
\usepackage[font={small,it}]{caption} % Gör figur-/tabelltexter små och italiserade
\usepackage{color} % Färger
\usepackage{graphicx} % För figurer
\usepackage{wrapfig} % För figurer
\usepackage{subfig} % För figurer
\usepackage{hyperref} % Hyperlinks
\usepackage{verbatim} % Fungerar som en kommentar i texten, där kommandon inte längre är kommandon utan bara text.
\usepackage[framed,numbered,autolinebreaks,useliterate]{mcode} %matlabkod
%\usepackage{natbib} % För referenser (bibliography management), men kommenterar för annars blir det error av någon anledning.
\usepackage[backend=biber, citestyle=nature]{biblatex} % Bestämmer hur referenserna ska se ut.
\addbibresource{references.bib} % Lägger till referensbibliotek från Mendeley.
\usepackage{multirow} % För tabeller
\usepackage[titletoc]{appendix} % Gör så att appendix hamnar i innehållsförteckningen.
\usepackage{gensymb} % För både text och math-mode
\usepackage{float} % \begin{figure/table}[H] - gör så att bilden eller tabellen hamnar exakt där du vill. Utan det paketet har många förut haft problem med var figuren eller tabellen hamnar i rapporten, när [ht!] användes.

% Ta bort kommentaren om ni inte vill ha indragna stycken.
\usepackage{parskip} % Tar bort indragna stycken

% Ta bort kommentaren om ni vill ta bort de röda strecken i innehållsförteckningen.
%\hypersetup{colorlinks, citecolor=black,
 %		 	filecolor=black, linkcolor=black,
 %   		urlcolor=black}

\newcommand{\rd}[1] {\ensuremath{\mathrm{d}#1}} % Rakt d i  math-mode för infinitesimaler
\newcommand{\id}[1] {\ensuremath{\,\rd{}#1}} % Samma som \rd fast med ett litet blanksteg for integraler
\newcommand{\HRule}{\rule{\linewidth}{0.5mm}}

\begin{document}

\begin{titlepage}

% \begin{center}
% \textsc{\LARGE \textsc{Chalmers University of Technology}}\\[1.5cm]
% \textsc{\Large Mekanik} \\[0.5cm]

% \HRule \\[0.5cm]
% \huge {Latex Introduktion} \\[0.4cm]
% \HRule \\[1.5cm]
% \begin{minipage}[t]{0.45\textwidth}
% \begin{flushleft} \large
% \emph{Författare:}\\
% \textsc{Emilsson}  Emilia \\
% emilia@student.chalmers.se \\
% \vspace{10 px}
% \end{flushleft}
% \end{minipage}
% \
% \begin{minipage}[t]{0.45\textwidth}
% \begin{flushleft} \large
% \emph{Handledare:} \\
% \textsc{Cramstedt} Julia \\
% \textsc{Kööhler} Mauritz \\
% \textsc{Svensson} Markus \\
% \end{flushleft}
% \end{minipage}
% \vfill
% {\large \today}
% \end{center}


\begin{center}

% Title och underrubriker
\Huge \textbf {\LaTeX-kurs
}\\[.2in]
        \Large
        

       \vspace{.25in}
       
       \small \emph{Arrangör}\\
       {\bf SnKfKb }\\[.2in]
       
       \vspace{1in}



% Vilka är handledarna 
\large \textbf{Handledare}  \normalsize \\
\begin{table}[h]
\centering
\begin{tabular}{lr}\hline \\
%Name & Email\\ \\ \hline \\
Mauritz Kööhler & maukoo@student.chalmers.se \\
Julia Cramstedt & juliacr@student.chalmers.se  \\
Markus Svensson & marksve@student.chalmers.se \\

\end{tabular}
\end{table}

% Vem har författat
\textbf{Författare} \normalsize \\
 
\begin{table}[h]
    \centering
    \begin{tabular}{lr}\hline \\
%Name & Email \\ \\ \hline
       Emilia Emilsson  & emilia@chalmers.se \\
    \end{tabular}
    \label{tab:my_label}
\end{table}


%Logga

\vspace{1in}
\begin{figure}[h]
    \centering
    \includegraphics[width=0.2\linewidth]{AvancezChalmers_black_centered.pdf}
\end{figure}
\vspace{.2in}

% Go avslutning
\centering{}
{\fontfamily{lmdh}\selectfont \Large{\textsc{Sektion: KfKb}}}\\
\normalsize
\textsc{Chalmers Tekniska Högskola}\\
\vspace{.2cm}
\today

\end{center}


\end{titlepage}

\newpage
\tableofcontents
\thispagestyle{empty}
\newpage

\setcounter{apge}{}
\section{Text}
Vill man ha en text kan man skriva den här! \\
Säger man något smart som måste citeras kan man göra det här \cite{Absorption}.

\section*{En rubrik utan nummer}
Vill man också ha en rubrik utan nummer kan kan bara lägga till en '*' när man skriver kommandot till att göra en rubrik.

\section{Matematiska grejer}
Det som är det snyggaste i \LaTeX \space är de matematiska finesserna! Ekvationer och matriser blir i min mening väldigt snygga och här nedan kommer några exempel.\\
\subsection{Ekvationer}

$\left \{ \begin{array}{ll}
 \text{Hitta }  U \in V_{h} \text{ sådan att} \\ \\
 \displaystyle\int_{0}^{L} \dot{U}\chi dx + c\int_{0}^{L} U'\chi \;dx + D\int_{0}^{L} U'\chi' \;dx + v(L)(k_{L}U(L,t) - cU(L,t)) + v(0)(k_{0}U(0,t) + cU(0,t)) \\ = \int_{0}^{L} f\chi \;dx \quad \forall  \chi \in V_{h}
\end{array} \right. $ \\ 

\begin{equation}
   \sum_{j=0}^{m+1} \Vec{\dot{\xi_{j}}}(t) \underbrace { \Bigl (\int_{0}^{L} \varphi_{j}\varphi_{i}\;dx \Bigr ) }_{M_{ij}} + c \biggl ( \sum_{j=0}^{m+1} \Vec{\xi_{j}}(t) \underbrace{ \Bigl ( \int_{0}^{L} \varphi_{j}'\varphi_{i}\;dx \Bigr ) }_{C_{ij}} \biggr ) +D \biggl ( \sum_{j=0}^{m+1} \Vec{\xi_{j}}(t) \underbrace{ \Bigl ( \int_{0}^{L} \varphi_{j}'\varphi_{i}'\;dx \Bigr ) }_{S_{ij}} \biggr )+
\end{equation}

\subsection{Matriser}

\[ M_{i,j}=M_{j,i}=\left[ \begin{array}{ccc}

\int_{0}^{L}\varphi_{0}^2 & \dots & \int_{0}^{L}\varphi_{m+1}\varphi_0 \\
\vdots & \ddots & \vdots \\
\int_{0}^{L}\varphi_{m+1}\varphi_{0} & \dots & \int_{0}^{L}\varphi_{m+1}^2

\end{array} \right]
=\frac{h}{6}
\left[ \begin{array}{ccccccc}

2 & 1 & 0 & \dots & \dots & \dots & 0 \\
1 & 4 & 1 &  &  &  & \vdots \\
0 & 1 & \ddots & \ddots &  &  & \vdots \\
\vdots &  & \ddots & \ddots & \ddots &  & \vdots \\
\vdots &  &  & \ddots & \ddots & 1 & 0 \\
\vdots &  &  &  & 1 & 4 & 1 \\
0 & \dots & \dots & \dots & 0 & 1 & 2 \\

\end{array} \right]_{(m+2)\times(m+2)} \]

\vfill
\section{Tabeller}
Tabeller i \LaTeX är väldigt stilrena och jag tycker de är väldigt anpassningsbara för det man vill göra! En tabell behöver inte vara vad man tänker sig är en tabell heller, en tabell i \LaTeX \space kan bara bygga en struktur där man tvingar in information på en viss plats! Lek lite med bokstäverna och strecken vid begin{tabular} och användandet av hline och se vad som händer!

\begin{table}[H]
    \centering
    \caption{Här är en ganska straight forward tabell som har streck mellan varje kolumn och alla kolumner är centrerade. Jag gillar att använda mig av två hline för att separera rubrikerna från värden. Jag har använt olika boksäver för att justera var texten lägger sig i sin cell.}
    \begin{tabular}{|r|c|l|} \hline \hline
       \textbf{Reactant} & \textbf{Quantity} [$\mu$l] & \textbf{Concentration} \\ \hline \hline
          $H_2O$ & 40.5 & - \\ \hline
         10x Dream Taq buffer & 5 & 1x \\ \hline
         GghA-His\_Dw2 & 1 & 0,2[$\mu$M] \\ \hline
         N220Y-primer & 1 & 0,2[$\mu$M] \\ \hline
         MolBio PLasmid & 1 & \\ \hline
         dNTP mix & 4 & 100 [mM] \\ \hline
         DreamTaq Polymeras & 0.5 & 25 [mU/$\mu$l] \\ \hline
         \textbf{Total volume:} & \textbf{53} & \\ \hline
    \end{tabular}
    \label{tab:PCR1}
\end{table}

\begin{table}[h!]
    \centering
    \caption{Här är en tabell där det inte finns några linjer alls. Det blir en helt annan uppfattning men strukturen finns där i bakgrunden. }
    \begin{tabular}{ccc}
    Step 1 & 94\degree & 3 min \\
    Step 2 & 94\degree & 30 sec \\
    Step 3 & 56\degree & 30 sec \\
    Step 4 & 72\degree & 30 sec \\
    Step 5 & Go to step 2 & Repeat x30 \\
    Step 6 & 72\degree & 2 min \\
    Step 4 & 4\degree & $\infty$
    \end{tabular}
    \label{tab:PCRP1}
\end{table}

\begin{table}[H]
    \centering
        \caption{Här är det mer tecke som gör att man behöver bygga upp omgivningar för specialtecken. Detta görs med \$ kring det man skriver.}
    \begin{tabular}{|c|c|c|c|c|} \hline \hline
         \textbf{Sample} &  \textbf{K$_m$ (mM)} & \textbf{Error K$_m$} & \textbf{V$_{max}$ ($\mu$M/min)} & \textbf{Error V$_{max}$}\\ \hline
         \textbf{Purified Wildtype} & 0.354156 $\pm$ 0.1127 & 31.82 \% & 2.69601 $\pm$ 0.169 & 6.267 \% \\ \hline 
         \textbf{Purified Mutant} & 1.34342 $\pm$ 1.333 & 99.25 \% & 5.65553 $\pm$ 1.187 & 20.99 \% \\ \hline \hline
    \end{tabular}
    \label{tab:KmVmax}
\end{table}

\subsection{Tabeller bredvid varandra}
Vill man ha två tabeller som ligger bredvid varandra får man tänka till lite. Här kommer ett sätt som jag har löst det på.

\begin{center}
    

\begin{table}[!htb]
    \begin{minipage}{.5\linewidth}
      \centering
        \begin{tabular}{l r r}
            \textbf{INTÄKTER} & \textbf{2021} & \textbf{2020} \\
            Program, Bt & 25 000 & 25 000 \\
            Program, Kf & 15 000 & 15 000 \\
            
        \end{tabular}
    \end{minipage}%
    \begin{minipage}{.5\linewidth}
      \centering
        \begin{tabular}{l r r}
            \textbf{UTGIFTER} & \textbf{2021} & \textbf{2020} \\
            bØf & -39 600 & -29 600 \\
            SnKfKb & -13 000 & -13 000\\
            KfIK & -11 500 & -11 500 \\
            
        \end{tabular}
    \end{minipage}
\end{table}

\end{center}

\section{Figurer}

\begin{figure}[H]
    \centering
    \includegraphics[width=0.85\linewidth]{photo.png}
    \caption{Summary of how the photosynthesis works in the chloroplasts of plants}
    \label{fig:Foto}
\end{figure}

\subsection{Subfigurer}
Vill man fortsätta med en underrubrik skriver man bara sub före section för kommandot!

\begin{figure}[H]
  \subfloat[][Michaelis-Menten plot for the purified wildtype samples.]{\includegraphics[width=.5\linewidth]{MMWT.png} \label{fig:MMWTexcel}}
  \quad
  \subfloat[][Michaelis-Menten plot for the purified wildtype samples made with the online tool at http://www.ic50.tk/kmvmax.html. The x- and y-axis represents the same variables and units as in the graph to the left. ]{\includegraphics[width=0.5\linewidth]{wildtypeplot2.jpg} 
  \label{fig:MMWTtool}}
\end{figure}

\newpage
\begin{thebibliography}{99}% börjar referenslistan
\bibitem{NE}Aquilonius S-M. Alzheimers sjukdom. I: Nationalencyklopedin [internet]. [15/10-2018]. 

\bibitem{Thies}Thies W, Bleiler L. 2011 Alzheimers Disease facts and figures. Elsevier Inc [Internet]. 2011 [citerad 22 November 2018]; volym (7:2):208–244

\bibitem{Heine} Heine, V M, Dooves S, Holmes D, \& Wagner J.
\emph{Induced Pluripotent Stem Cells in Brain Diseases.} Amsterdam: Springer; 2012

\bibitem{Yildirim}Yildrim S. \emph{Induced Pluripotent Stem cells}. Konya: Springer; 2012

\bibitem{Robbins} Robbins, J P, \& Price, J. (2017)
\emph{Human Induced Pluripotent Stem Cells As Research Tool In Alzheimer's Disease.} Psychological medicine, 47 (15), 2587-2592

\end{thebibliography} %slutar referenslistan


\printbibliography

\end{document}